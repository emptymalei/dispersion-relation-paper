% ****** Start of file apssamp.tex ******
%
%   This file is part of the APS files in the REVTeX 4.1 distribution.
%   Version 4.1r of REVTeX, August 2010
%
%   Copyright (c) 2009, 2010 The American Physical Society.
%
%   See the REVTeX 4 README file for restrictions and more information.
%
% TeX'ing this file requires that you have AMS-LaTeX 2.0 installed
% as well as the rest of the prerequisites for REVTeX 4.1
%
% See the REVTeX 4 README file
% It also requires running BibTeX. The commands are as follows:
%
%  1)  latex apssamp.tex
%  2)  bibtex apssamp
%  3)  latex apssamp.tex
%  4)  latex apssamp.tex
%
\documentclass[%
 reprint,
%superscriptaddress,
%groupedaddress,
%unsortedaddress,
%runinaddress,
%frontmatterverbose,
%preprint,
%showpacs,preprintnumbers,
%nofootinbib,
%nobibnotes,
%bibnotes,
 amsmath,amssymb,
 aps,
 prd
%pra,
%prb,
%rmp,
%prstab,
%prstper,
%floatfix,
]{revtex4-1}

\usepackage{graphicx}% Include figure files
\usepackage{dcolumn}% Align table columns on decimal point
\usepackage{bm}% bold math
%\usepackage{hyperref}% add hypertext capabilities
%\usepackage[mathlines]{lineno}% Enable numbering of text and display math
%\linenumbers\relax % Commence numbering lines

%\usepackage[showframe,%Uncomment any one of the following lines to test
%%scale=0.7, marginratio={1:1, 2:3}, ignoreall,% default settings
%%text={7in,10in},centering,
%%margin=1.5in,
%%total={6.5in,8.75in}, top=1.2in, left=0.9in, includefoot,
%%height=10in,a5paper,hmargin={3cm,0.8in},
%]{geometry}

\newcommand{\ii}{\mathrm i}


%%%%%%%%%%%%%%%%%%%%%%%%%%%%%%%%%
%%%%% Packages for draft only
%%%%%%%%%%%%%%%%%%%%%%%%%%%%%%%%%%

\usepackage[normalem]{ulem}
\usepackage{xcolor}


\begin{document}

%\preprint{APS/123-QED}

\title{Dispersion Relation and Neutrino Flavor Instabilities in Fast Modes}% Force line breaks with \\
\thanks{A footnote to the article title}%

\author{Ann Author}
 \altaffiliation[Also at ]{Physics Department, XYZ University.}%Lines break automatically or can be forced with \\
\author{Second Author}%
 \email{Second.Author@institution.edu}
\affiliation{%
 Authors' institution and/or address\\
 This line break forced with \textbackslash\textbackslash
}%


\date{\today}% It is always \today, today,
             %  but any date may be explicitly specified

\begin{abstract}

ABSTRACT PLACEHOLDER


% \begin{description}
% \item[Usage]
% Secondary publications and information retrieval purposes.
% \item[PACS numbers]
% May be entered using the \verb+\pacs{#1}+ command.
% \item[Structure]
% You may use the \texttt{description} environment to structure your abstract;
% use the optional argument of the \verb+\item+ command to give the category of each item.
% \end{description}
\end{abstract}

% \pacs{Valid PACS appear here}% PACS, the Physics and Astronomy
                             % Classification Scheme.
%\keywords{Suggested keywords}%Use showkeys class option if keyword
                              %display desired
\maketitle

%\tableofcontents



\section{\label{sec-introduction}Introduction}


Neutrino flavor conversions in vacuum are linear effects in Schrodinger equation. In dense neutrino media, neutrinos demonstrate highly nonlinear flavor transformations due to forward scattering interactions of neutrinos. Such interactions lead to flavor conversions of different category from vacuum oscillations. The technique used to investigate the nonlinear effect is linear stability analysis.\cite{Banerjee2011a,Raffelt2013} Recent studies by I. Izaguirre, G. Raffelt, and I. Tamborra show that neutrino flavor conversion instabilities is related to gaps in dispersion relation.\cite{Izaguirre2016a} They showed that dispersion relation can be defined and calculated in linear regime of neutrino flavor conversions. We argue that neutrino flavor conversions instabilities are not exactly mapped to gaps in dispersion relation for continuous angular distributions of neutrino emissions. In Sec. \ref{sec-dr} we review the formalism of linear stability analysis and dispersion relation.


\section{\label{sec-dr}Dispersion Relation of Neutrino Flavor Conversion}

We consider two-flavor scenario of neutrino oscillations. As an initial condition, all neutrinos and antineutrinos are emitted approximately as electron flavors. For the purpose of linear stability analysis, the single particle density matrix for neutrinos is explicitly defined as
\begin{equation}
   \rho = \frac{1}{2} \begin{pmatrix}
   1 & \epsilon \\
   \epsilon^* & -1
\end{pmatrix}.
\end{equation}
To determine the flavor evolution, Liouville equation of neutrinos is used,
\begin{equation}
\ii (\partial_t + \mathbf v\cdot \boldsymbol{\nabla}) \rho = \left[ H, \rho \right],
\label{eqn-liouville-eqn}
\end{equation}
where $H$ is the Hamiltonian for neutrino oscillations. Density matrix and equation of motion for antineutrinos are defined in the same maner with the corresponding Hamiltonian for antineutrinos.

In principle, neutrino oscillations Hamiltonian depend on three different contributions, vacuum oscillations $H_{\mathrm v}$, interactions with matter $H_{\mathrm m}$, and interactions with neutrinos themselves $H_{\nu\nu}$. The concentration of this work is on fast neutrino oscillations, which would occur even without neutrino mass differences. However, the vacuum term can always be combined with matter term by redefining new matter potential. Thus neglecting vacuum term doesn't change the formalism of linear stability analysis. Vacuum oscillations term is defined
\begin{equation}
   H_{\mathrm v} = -\frac{\omega_{\mathrm v}}{2} \sigma_3,
\end{equation}
where $\omega = \frac{\delta m^2}{2}$ with $\delta m^2$ being the mass squared difference in this two flavor scenario. Interactions with matter is described by matter potential
\begin{equation}
   H_{\mathrm m} = \frac{1}{2}\lambda \sigma_3.
\end{equation}
where $\lambda = \sqrt{2}G_{\mathrm F} n_{\mathrm e}$. $G_{\mathrm F}$ is the Fermi constant and $n_{\mathrm e}$ is the number density of electrons in the background. In order to calculate the neutrino forward scattering, the spectrum of neutrino (antineutrinos) distributions $f_{\nu_{\mathrm e}(\bar \nu_{\mathrm e})}(p)$, where $p$ is the four momentum of neutrinos (antineutrinos), is required. Following the definition of electron lepton number $G(\hat{\mathbf v})$ in \onlinecite{Izaguirre2016a}, neutrino forward scattering potential is
\begin{equation}
H_{\nu\nu} = \sqrt{2} G_{\mathrm F} \iint \frac{\mathrm d \cos\theta' \mathrm d\phi'}{4\pi} G(\cos\theta',\phi') v^\mu v'_\mu \rho,
\end{equation}
where $v^\mu = ( 1, \sin\theta\cos\phi, \sin\theta\sin\phi, \cos\theta )^{\mathrm T}$ in spherical coordinate system.

Linear stability analysis of Eq. \eqref{eqn-liouville-eqn} for axial symmetric neutrino emission shows that
\begin{align}
&\det \left( \omega \mathrm{I} + \frac{1}{2}
\begin{pmatrix}
   I_0 & 0 & 0 & -I_1 \\
   0 & -\frac{1}{2} (I_0 - I_2) & 0 & 0 \\
   0 & 0 & -\frac{1}{2} (I_0 - I_2) & 0 \\
   I_1 & 0 & 0 & -I_2
\end{pmatrix}\right) \nonumber\\
&=0,
\label{eqn-det-polarization-tensor}
\end{align}
where
\begin{equation}
   I_m(\theta)=\int_{\cos\theta_H}^{\cos\theta_L} d\cos\theta G(\theta) \frac{\cos^m\theta}{1 -  \left(\lvert k\rvert /\omega\right) \cos\theta }.
\end{equation}
For convinience, we denote $u=\cos\theta$ later on. Two categories of solutions was found, namely the multi-azimuthal angle (MAA) solution and multi-zenith angle (MZA) solution. The MAA solution is only related to instabilities due to azimuthal angle. Its formal solution is
\begin{equation}
   \omega = \frac{1}{4}(I_0 - I_2).
\end{equation}
The MZA solution is related to both azimuthal angle and zenith angle, which has formal solutions
\begin{equation}
\omega = - \frac{1}{4} \left( I_0 - I_2 \pm \sqrt{ (I_0 + I_2 - 2 I_1) (I_0 + I_2 + 2 I_1) } \right).
\end{equation}


Eq. \eqref{eqn-det-polarization-tensor} is equivalent to dispersion relation defined in \onlinecite{Izaguirre2016a}.

%%%%%%%%%%%%%%%%%
%% To BE Added
%%%%%%%%%%%%%%%%
{\color{red}{\bf HAVE TO EXPLAIN THE IDEA OF GAP AND INSTABILITY HERE.}}






\section{\label{sec-twobeams}Discrete Zenith Angle Model}

Axially symmetric two emission angles model is determined by the electron lepton number
\begin{equation}
G(u)= \sum_{i=1}^2 g_i \delta(u - u_i).
\end{equation}

The MAA solution is an equation of hypobola for $\omega$ and $k$, which as asymptotic lines $\omega = k u_i$ for $i=1,2$. The hyperbola equation has two solutions for $\omega$ given real $k$ or two solutions for $k$ given real $\omega$. The solutions are either real which indicates stable solutions or complex which indicates exponential growth in linear regime. Thus the equivalence of gap and instabilities is guaranteed.

However, this conclusion can not be generalized to arbitrary number of emission angles. As an example, we calculate the three emission angles configuration.

{\color{red}\bf PLACE THE PLOTS HERE}


\section{\label{sec-continuous-spectrum}Continuous Spectrum}



\begin{equation}
   4 = \bar I_0 - \bar I_2
\end{equation}

\begin{equation}
   4 = \frac{1}{k} \int G(u) \frac{ 1 - u^2 }{ \omega/k - u }
\end{equation}

\begin{align}
k_R =& \frac{1}{4}\left(  \mathscr P \int G(u) \frac{ 1 - u^2 }{  - u }  \right) \\
k_I =&  \frac{\pi}{4}G(0) \operatorname{Sign}\left( \omega \right) \operatorname{Sign}\left(  \operatorname{Im}(k)  \right)
\end{align}


\begin{equation}
   \lvert k_I \rvert  =  \frac{\pi}{4}G(0) \operatorname{Sign}\left( \omega \right).
\end{equation}


\section{\label{sec-conclusion}Conclusion}






















%%%%%%%%%%%%%%%%%%%%%%%%%%%%%%%%%%%%%%%%%%%%%
%% APPENDIX
%%%%%%%%%%%%%%%%%%%%%%%%%%%%%%%%%%%%%%%%%%%%%

\appendix
\section{\label{sec-outline}Plan of the paper}



\begin{itemize}
    \item \sout{Review fast mode oscillations}
    \item State what has been done in Raffelt's paper.
    \item The conclusion is not true.
    \item Two beams examples to prove that the number of solutions is the key.
    \item Show that the continuous case is not related to gap at all. Box spectrum?
    \item Continuous spectrum or Garching group, data to show that we can prove the location of the instability.
\end{itemize}


But I have a question. Is it really reliable? Should I use principal value integral for box spectrum DR?

We are still not crystal clear about the relation between gap and lsa.





\bibliographystyle{apsrev4-1}
\bibliography{ref.bib}

\end{document}
%
% ****** End of file apssamp.tex ******
